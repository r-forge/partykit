
%***********************************************************************

% This is a template to be used for the preparation of
% papers submitted to the 34th International Workshop on
% Statistical Modelling, to be held in Guimar?es, Portugal,
% July 7-12, 2019.

% Please follow the following general guidelines:
%
% (1) Do not specify any definitions, commands or style parameters.
%     Upon submission, your file will be checked for presence of
%     \newcommand or \def statements and if found, error message will be reported
%     by the submission form.
%
% (2) Follow the template below very tightly.
%
% (3) Include .pdf figures using the \includegraphics
%      command, an example of which are given below.
%
% (4) Use file names which begin with the surname of the first author.
%
% (5) When creating labels for cross-references, please start always
%     by surname of the first author, e.g., \label{smith:likelihood}
%
% (6) The template below contains some example materials
%      to guide you through the preparation of your paper.  However,
%      remove all the redundant material from your final document
%      before submitting.

% The guidelines above are needed in order to be able to combine all
% the papers into a single proceedings book of acceptable quality.
% Please follow the guidelines as strictly as possible. Deviations may
% result in papers being either refused by the registration form
% or sent back to the authors with the request to change
% their documents according to the guidelines.

% Special characters:
% Please do not use special characters (e.g., accents).
% Use TeX composition instead, such as \~n, \'a, \`e, \v{s}, \r{u} etc.

% Changes as of IWSM 2013:
%  * \usepackage{booktabs} added which allows \toprule et al. in the tabular environment
%    (\hline\hline is not longer used)
%  * '^\T' added in iwsm.sty to denote transposed vectors and matrices within math (see example below)
%  * \usepackage{amsmath, amssymb} introduced since IWSM 2012 is allowed (allowing usage of boldsymbols
%    and other handy constructions (align, pmatrix etc.) within math)
%  * \usepackage{psfrag} introduced since IWSM 2012 is NOT allowed
%
%

%***********************************************************************
% PLEASE LEAVE THIS PART UNCHANGED
%***********************************************************************

\documentclass[twoside]{report}
\usepackage{iwsm}
\usepackage{graphicx}
\usepackage{amsmath, amssymb}
\usepackage{booktabs}

% Please do not specify any new definitions, any new commands,
% and do not specify any style parameters.
% The preamble of the document should be left unchanged.

\begin{document}

%***********************************************************************
% PLEASE INSERT YOUR CONTENT FROM HERE
%***********************************************************************

% Title and running title to be used as left header:
\title{Distributional Trees for Circular Data}
\titlerunning{Distributional Trees for Circular Data}

% Authors and running list of authors to be used as right header:
\author{Lisa Schlosser\inst{1}, 
Moritz N. Lang\inst{1},
Torsten Hothorn\inst{2},
Georg J. Mayr\inst{1},
Reto Stauffer\inst{1},
Achim Zeileis\inst{1}}
\authorrunning{Schlosser et al.}    %% use \authorrunning{Surname 1} if only 1 author
                                    %% use \authorrunning{Surname 1 and Surname2} if two authors
                                    %% use \authorrunning{Surname 1 et al.} if more than two authors

% Institutes of all authors
% Include city and country of each institute, do not include the full address.
\institute{University of Innsbruck, Austria
\and University of Zurich, Switzerland}

% E-mail of presenting author for correspondence
\email{Lisa.Schlosser@uibk.ac.at}

% Brief abstract of the paper:
\abstract{
%Circular response variables occur in a variety of applications, e.g., when
%geographical directions are considered or measurements are taken on a $24$-hour scale.
A commonly chosen distribution for modeling circular data is the von Mises distribution.
As an alternative to estimating its distribution parameters by generalized additive models 
we propose the application of distributional trees which are capable of detecting also 
non-additive effects, select covariates and their possible interactions automatically and 
provide easily interpretable models allowing for a wide range of inference procedures. 
This approach is illustrated by an application of distributional trees on hourly wind 
direction forecasts with a set of meteorological covariates, all measured at the airport 
of Innsbruck, Austria.
}


% Keywords (at most 5):
\keywords{Distributional trees; Circular response; Von Mises distribution.}

% Produce the title:
\maketitle

%***********************************************************************

% Sections and subsections (do not use lower levels):

%\section{Motivation}
%Circular response variables occur in a variety of applications and subject areas. 
%For example, gun crime data on a $24$-hour scale is analysed in the social-economics, 
%animal orientation or gene-structure analysis are often subject of examination in biology, 
%and wind direction is one of the most important weather variables in meteorology.
%Representing such a two-dimensional direction by an angle in a unity circle requires 
%the specification of a starting point (angle 0) and either a clockwise or anti-clockwise 
%rotation. However, these settings can vary but should not influence statistical 
%inference from a fitted model.
%For that reason circular data demand for purpose-built statistical methods. 
%A commonly chosen distribution for modeling circular data is the von Mises distribution.
%Estimating its two distribution parameters by generalized additive models using a set of 
%regressor variables is one possible approach that can be successful in various settings.
%As an alternative we propose the application of distributional trees 
%which are capable of detecting also non-additive effects and provide easily interpretable 
%models allowing for a wide range of inference procedures.
%Moreover, covariates and their possible interactions are selected automatically
%such that no expert knowledge is required to specify the model in advance.

\section{Methodology}

%\subsection{Von Mises Distribution}
%The von Mises distribution is also known as the circular normal distribution 
%and its density function is given by
%\begin{equation}
%  f_{vM}(y; \mu, \kappa) = \frac{1}{2 \pi I_0(\kappa)}~e^{ \kappa \cos(y - \mu)}\label{equ:vm}
%\end{equation}
%with a location parameter $\mu$, a conentration parameter $\kappa$, and $I_0(\kappa)$ being the 
%modified Bessel function of the first kind and order $0$.
%For $\kappa=0$ a uniform distribution is obtained. Otherwise, a large value of $\kappa$ indicates
%a high concentration of the distribution around $\mu$. Hence, $\frac{1}{\kappa}$ is analogous
%to the variance $\sigma^2$ of the normal distribution. 
%See e.g., Jammalamadaka and Sengupta (2001) or Mardia and Jupp (2009) for a more detailed overview.


%\subsection{Distributional Trees}
Distributional trees~(Schlosser et. al, 2018) embed distributional modeling into the framework 
of regression trees.
%Distributional trees~(Schlosser et. al, 2018) embed distributional modeling such as
%provided by generalized additive models for location, scale, and shape~(GAMLSS, 
%Rigby and Stasinopoulos, 2005) into the framework of regression trees~(Breiman et. al, 1984).
The basic idea is to partition the covariate space recursively into subgroups 
such that a homogeneous distributional model can be fit to the response in each resulting subgroup.
In particular, one of the unbiased recursive partitioning algorithms MOB~(Zeileis et. al, 2008) 
and CTree~(Hothorn et. al, 2006) is applied and in each node the parameters of a specified 
distribution family are estimated via maximum likelihood.
For each split the split variable is selected based on $p$-values from permutation or M-fluctuation 
tests evaluating associations between model scores and each possible split variable. After choosing 
the split variable showing the highest association to the goodness of fit of the model the corresponding split point is selected in a separate step, either by optimizing the log-likelihood function (MOB)
or by evaluating a linear two-sample statistic (CTree) over all possible partitions.

In this study distributional trees are applied to circular response data employing the MOB algorithm
and the von Mises distribution which is also known as the circular 
normal distribution. Its density function is given by
\begin{equation}
  f_{vM}(y; \mu, \kappa) = \frac{1}{2 \pi I_0(\kappa)}~e^{ \kappa \cos(y - \mu)}\label{equ:vm}
\end{equation}
with location parameter $\mu$, conentration parameter $\kappa$, and $I_0(\kappa)$ being the 
modified Bessel function of the first kind and order $0$
%For $\kappa=0$ a uniform distribution is obtained. Otherwise, a large value of $\kappa$ indicates
%a high concentration of the distribution around $\mu$. Hence, $\frac{1}{\kappa}$ is analogous
%to the variance $\sigma^2$ of the normal distribution. 
(see, e.g., Jammalamadaka and Sengupta (2001) for a more detailed overview).

In each node of the tree the distribution parameters $\mu$ and $\kappa$ are 
%vector $\theta = (\mu,\kappa)$ is 
estimated by maximizing $l(y, \mu, \kappa) = log(f_{vM}(y;\mu, \kappa))$. 
For the resulting estimators $\hat{\mu}$ and $\hat{\kappa}$ model scores are calculated by  
%$s(y,\hat{\mu}, \hat{\kappa}) = \left(\frac{\partial l(y, \hat{\mu}, \hat{\kappa})}{\partial \mu},
%\frac{\partial l(y, \hat{\mu}, \hat{\kappa})}{\partial \kappa}\right)$
$s(y,\hat{\mu}, \hat{\kappa}) = (\partial_{\mu} l(y, \hat{\mu}, \hat{\kappa}),
\partial_{\kappa}l(y, \hat{\mu}, \hat{\kappa}))$
yielding a $n \times 2$-matrix for a data set of $n$ observations.
%$s(y,\hat{\theta}) = \frac{\partial l(y, \hat{\theta})}{\partial \theta}$. 
In that way for each parameter a value indicating the goodness of fit with respect to one single 
observation is provided such that changes can be detected not only in the location parameter $\mu$ 
but also in the concentration parameter $\kappa$.


%For each possible split variable $Z$ the association between the scores and $Z$ is evaluated by a permutation or M-fluctuation test such that the resulting $p$-values allow for an unbiased split variable selection on a unified scale. The variable showing the highest association to the scores is chosen as a split variable. The corresponding split point is then selected based on a greedy search over all possible split points. After splitting the data into subgroups the procedure is repeated in each of the resulting child nodes until eather a stopping criteria is reached or only equal observations are left in a node. 



\section{Application}

\begin{figure}[!ht]\centering
\includegraphics[height = .5\textheight,angle=90,origin=c]{circtree_plot_big.pdf}
\caption{Fitted distribution tree for a circular response employing the von Mises distribution.}
\label{fig:tree}
\end{figure}

Wind is a classical circular quantity and accurate forecasts of wind direction
are of great importance for decision making processes and risk management,
e.g., in air traffic management or renewable energy production. This study
presents an application of circular regression trees  on hourly direction
forecasts at Innsbruck airport.  Innsbruck lies at the bottom of the Inn
Valley, a deep mountain valley in the Eastern European Alps orientated parallel
to the Alpine crest. At Innsbruck, the west-east aligned Inn Valley is bordered
in the North with a approx. 2300\,m high mountain ridge and opens to the South
into the crest-perpendicular Wipp Valley. As potential split variables we
employ several 1-hourly lagged observed weather variables (e.g., wind
direction, wind speed, relative humidity, pressure gradients along the valley)
and derived spatial and temporal difference of these lagged observations. 

As an illustration of the tree based wind direction forecasts at Innsbruck
airport, a BIC based post-pruned regression tree of depth three is shown in
Figure~\ref{fig:tree}. Three distinct wind regimes are fitted within the final
nodes: Firstly, up-valley winds blowing from the valley mouth towards the upper
valley (from east to west, nodes 4 and 5). Secondly, downslope winds blowing
across the Alpine crest along the Wipp Valley towards Innsbruck (from
south-east to north-west, nodes 7 and 8). Thirdly, down-valley winds blowing
in the direction of the valley mouth (from west to east, nodes 12, 14, and 15).
The final node 11 with a concentration parameter $\kappa$ of only $0.84$
captures the cases with rather low wind speeds which typically don't show a
clear wind direction or distinct wind regime. In general, winds typically blow
from areas of high pressure to areas of low pressure where strong pressure
gradients yield high wind speeds and a rather well-defined wind direction.
Thus, the splits in the lagged wind direction perform a clustering into the
different wind regimes and the additional splits in pressure gradients or wind
speed divide the predictions according to the expected consistency of the wind
direction. This can be seen in the final nodes with strong pressure gradients
or high wind speeds where most training cases are classified correctly and
therefore the concentration parameter as an indicator of the certainty of the
forecast is rather large. 

\section{Discussion / Summary and Outlook}
This study shows that applying distributional trees on circular data provides easily 
interpretable models clearly illustrating the impact of given covariates and their possible 
interactions on the distribution of the dependent variable. Moreover, covariates are selected
automatically such that no prespecification of the model is required which can be a challenging 
task, particularly for a meteorlogical quantity such as wind direction with high variation and 
many possible influencing factors.

Based on the results it can be assumed that including covariates such as lagged observations of 
wind direction as linear regressors provides valuable information. 
However, in order to properly account not only for the circular properties of the response but 
also of regressors, additional adaptions of distributional trees are required. In particular, 
if the selected split variable is a circular variable, e.g. wind direction, this demands for finding
two split point defining a ``split interval'' instead of a single split point.
This is also supported by the results illustrated in Figure~\ref{fig:tree} as mutliple splits in 
the linearly employed regressor \textit{dd} (lagged wind direction) are required in order to 
cluster the training data in corresponding wind directions.  

%While the application of the von Mises distribution allows for modeling a circular response variable,
%additional adaptions have to be made when including circular regressor variables in a distributional tree
%model. In particular, if the selected split variable is a circular variable this demands for finding
%not only one split point but a ``split interval''. This consideration is of interest for many applications,
%for example when modeling wind directions it might be desirable to include day time or earlier 
%observations of wind directions as regressor variables.

Another possible extension of the method as presented is to build an ensemble of 
distributional trees to regularize and stabilize the model 
(see, e.g., distributional forests, Schlosser et. al, 2018). 
%In that way, it should not have a 
%significant impact on the results whether circular covariates are included as linear regressors
%yielding a larger number of splits or as circular regressors with ``intervall splits''.



\bigskip


%***********************************************************************

% Tables can be included at any place in the text.
% As general format, use tables with horizontal lines only
% (i.e., no vertical lines separating columns).
% Start and end each table with a double horizontal line.
% Tables are incorporated using a table environment:

%\begin{table}[!ht]\centering
% A caption is put above the table and a label is defined
% to be used for reference to this specific table.
% Use labels which are very unlikely to be used by
% other contributors; for example, use labels starting
% with the surname of the first author.
%\caption{\label{smith:tab1} Caption text \textbf{ABOVE} the table.}
% A table with three columns, where the first is left aligned,
% the second has centred entries, and the third is right aligned,
% is generated as follows (note: if requested, use \cmidrule{} instead of \cline{})
%\medskip
%\begin{tabular}{lcr}
% First line:
%\toprule[0.09 em]
% The body of the table:
%Title col1 & Title col2 & Title col3 \\
%\midrule
%row1 col1  & row1 col2  & row1 col3  \\
%row2 col1  & row2 col2  & row2 col3  \\ %
%row3 col1  & row3 col2  & row3 col3  \\
% last line:
%\bottomrule[0.09 em]
%\end{tabular}
%\end{table}

% In the text, reference to the Table can be made as follows:
%We refer to Table~\ref{smith:tab1} for a summary of our main results. Have a look to Table~\ref{smith:tab2} for
%an additional example.

%\begin{table}[!ht]\centering
%\caption{\label{smith:tab2} Caption text \textbf{ABOVE} the table.}
%\medskip
%\begin{tabular}{lcr}
% First line:
%\toprule[0.09 em]
% The body of the table:
%  &\multicolumn{2}{c}{Title  for cols 2 -3} \\
%\cmidrule{2-3} %
%Title1 & Title2 & Title3 \\
%\midrule
%& $a$  & $c$ \\
%& $b$  & $d$ \\ %
%\midrule[0 em]
%Total  & $a+b$  & $n$  \\
% last line:
%\bottomrule[0.09 em]
%\end{tabular}
%\end{table}



%***********************************************************************

% Figures can be included at any place in the text.
% The only allowed formats for figures are pdf files.
%
% Please, do not include figures in any other format.
%
% Use file names which are very unlikely to be used by
% other contributors; for example, use file names starting
% with the surname of the first author.
% Figures are incorporated using a figure environment:
% Make sure you specify the extension of the file (pdf)

%Finally a figure (in \verb|.pdf|!)

%\begin{figure}[!ht]\centering
% You can pre-specify the width of the graph:
%\includegraphics[width=8cm]{circmax-004.pdf}
% Below the figure, a caption is put, and a label is defined
% to be used for reference to this specific figure.
% Use labels which are very unlikely to be used by
% other contributors; for example, use labels starting
% with the surname of the first author.
%\caption{\label{fig:circtree} Fitted distribution tree for a circular response employing the von Mises distribution.}
%\end{figure}


% In the text, reference to the Figure can be made as follows:
%We refer to Figure~\ref{smith:fig1} for a~graphical representation.


%***********************************************************************

% Acknowledgments, if needed:
%\acknowledgments{Special Thanks to ... }

%***********************************************************************

% References should be placed in the text in author (year) form.
% The list of references should be placed below IN ALPHABETICAL ORDER.
% (Please follow the format of the examples very tightly).

\references
\begin{description}

%\item[Breiman, L., Friedman, J. H., Olshen, R. A., and Stone, C. J.] (1984).
%     Classification and Regression Trees.
%     {\it Wadsworth}.

\item[Hothorn, T., Hornik, K., and Zeileis, A.] (2006).
     Unbiased Recursive Partitioning: A Conditional Inference Framework.
     {\it Journal of Computational and Graphical Statistics}, {\bf 15}, 3,
     651\,--\,674. 

\item[Jammalamadaka, S. R., and Sengupta, A.] (2001).
     Topics in Circular Statistics.
     {\it World Scientific}, {\bf 5}. 

%\item[Mardia, K. V., and Jupp, P. E.] (2009).
%     Directional Statistics.
%     {\it John Wiley \& Sons}, {\bf 494}. 

%\item[Rigby, R. A., and Stasinopoulos, D. M.] (2005).
%     Generalized Additive Models for Location Scale and Shape (with Discussion).
%     {\it Journal of the Royal Statistical Society C}, {\bf 54}, 3,
%     507\,--\,554.

\item[Schlosser, L., Hothorn, T., and Zeileis, A.] (2018).
     Distributional Regression Forests for Probabilistic Precipitation Forecasting in Complex Terrain.
     {\it arXiv.org E-Print Archive}, https://arxiv.org/abs/1804.02921v2.

\item[Zeileis, A., Hothorn, T., and Hornik, K.] (2008).
     Model-Based Recursive Partitioning.
     {\it Journal of Computational and Graphical Statistics}, {\bf 17}, 2,
     492\,--\,514. 
     

\end{description}

\end{document}
