
%***********************************************************************

% This is a template to be used for the preparation of
% papers submitted to the 34th International Workshop on
% Statistical Modelling, to be held in Guimar?es, Portugal,
% July 7-12, 2019.

% Please follow the following general guidelines:
%
% (1) Do not specify any definitions, commands or style parameters.
%     Upon submission, your file will be checked for presence of
%     \newcommand or \def statements and if found, error message will be reported
%     by the submission form.
%
% (2) Follow the template below very tightly.
%
% (3) Include .pdf figures using the \includegraphics
%      command, an example of which are given below.
%
% (4) Use file names which begin with the surname of the first author.
%
% (5) When creating labels for cross-references, please start always
%     by surname of the first author, e.g., \label{smith:likelihood}
%
% (6) The template below contains some example materials
%      to guide you through the preparation of your paper.  However,
%      remove all the redundant material from your final document
%      before submitting.

% The guidelines above are needed in order to be able to combine all
% the papers into a single proceedings book of acceptable quality.
% Please follow the guidelines as strictly as possible. Deviations may
% result in papers being either refused by the registration form
% or sent back to the authors with the request to change
% their documents according to the guidelines.

% Special characters:
% Please do not use special characters (e.g., accents).
% Use TeX composition instead, such as \~n, \'a, \`e, \v{s}, \r{u} etc.

% Changes as of IWSM 2013:
%  * \usepackage{booktabs} added which allows \toprule et al. in the tabular environment
%    (\hline\hline is not longer used)
%  * '^\T' added in iwsm.sty to denote transposed vectors and matrices within math (see example below)
%  * \usepackage{amsmath, amssymb} introduced since IWSM 2012 is allowed (allowing usage of boldsymbols
%    and other handy constructions (align, pmatrix etc.) within math)
%  * \usepackage{psfrag} introduced since IWSM 2012 is NOT allowed
%
%

%***********************************************************************
% PLEASE LEAVE THIS PART UNCHANGED
%***********************************************************************

\documentclass[twoside]{report}
\usepackage{iwsm}
\usepackage{graphicx}
\usepackage{amsmath, amssymb}
\usepackage{booktabs}

% Please do not specify any new definitions, any new commands,
% and do not specify any style parameters.
% The preamble of the document should be left unchanged.

\begin{document}

%***********************************************************************
% PLEASE INSERT YOUR CONTENT FROM HERE
%***********************************************************************

% Title and running title to be used as left header:
\title{Distributional Trees for Circular Data}
\titlerunning{Distributional Trees for Circular Data}

% Authors and running list of authors to be used as right header:
\author{Lisa Schlosser\inst{1}, 
Moritz N. Lang\inst{1},
Torsten Hothorn\inst{2},
Georg J. Mayr\inst{1},
Reto Stauffer\inst{1},
Achim Zeileis\inst{1}}
\authorrunning{Schlosser et al.}    %% use \authorrunning{Surname 1} if only 1 author
                                    %% use \authorrunning{Surname 1 and Surname2} if two authors
                                    %% use \authorrunning{Surname 1 et al.} if more than two authors

% Institutes of all authors
% Include city and country of each institute, do not include the full address.
\institute{Universit\"at Innsbruck, Austria
\and Universit\"at Z\"urich, Switzerland}

% E-mail of presenting author for correspondence
\email{Lisa.Schlosser@uibk.ac.at}

% Brief abstract of the paper:
\abstract{
For probabilistic modeling of circular data the von Mises distribution is widely used.
To capture how its parameters change with covariates, a regression tree model
is proposed as an alternative to more commonly-used additive models. The resulting
distributional trees are easy to interpret, can detect non-additive effects, and select
covariates along with interactions automatically. For illustration, hourly wind 
direction forecasts are obtained at Innsbruck Airport based on a set of meteorological
measurements.
}


% Keywords (at most 5):
\keywords{Distributional trees; Circular response; Von Mises distribution.}

% Produce the title:
\maketitle

%***********************************************************************

% Sections and subsections (do not use lower levels):

%\section{Motivation}
%Circular response variables occur in a variety of applications and subject areas. 
%For example, gun crime data on a $24$-hour scale is analysed in the social-economics, 
%animal orientation or gene-structure analysis are often subject of examination in biology, 
%and wind direction is one of the most important weather variables in meteorology.
%Representing such a two-dimensional direction by an angle in a unity circle requires 
%the specification of a starting point (angle 0) and either a clockwise or anti-clockwise 
%rotation. However, these settings can vary but should not influence statistical 
%inference from a fitted model.
%For that reason circular data demand for purpose-built statistical methods. 
%A commonly chosen distribution for modeling circular data is the von Mises distribution.
%Estimating its two distribution parameters by generalized additive models using a set of 
%regressor variables is one possible approach that can be successful in various settings.
%As an alternative we propose the application of distributional trees 
%which are capable of detecting also non-additive effects and provide easily interpretable 
%models allowing for a wide range of inference procedures.
%Moreover, covariates and their possible interactions are selected automatically
%such that no expert knowledge is required to specify the model in advance.

\section{Methodology}

%\subsection{Von Mises Distribution}
%The von Mises distribution is also known as the circular normal distribution 
%and its density function is given by
%\begin{equation}
%  f_\mathrm{vM}(y; \mu, \kappa) = \frac{1}{2 \pi I_0(\kappa)}~e^{ \kappa \cos(y - \mu)}\label{equ:vm}
%\end{equation}
%with a location parameter $\mu$, a conentration parameter $\kappa$, and $I_0(\kappa)$ being the 
%modified Bessel function of the first kind and order $0$.
%For $\kappa=0$ a uniform distribution is obtained. Otherwise, a large value of $\kappa$ indicates
%a high concentration of the distribution around $\mu$. Hence, $\frac{1}{\kappa}$ is analogous
%to the variance $\sigma^2$ of the normal distribution. 
%See e.g., Jammalamadaka and Sengupta (2001) or Mardia and Jupp (2009) for a more detailed overview.


Distributional trees~(Schlosser et.~al, 2018) fuse distributional regression modeling with
regression trees based on the unbiased recursive partitioning algorithms MOB~(Zeileis et.~al, 2008)
or CTree~(Hothorn et.~al, 2006). The basic idea is to partition the covariate space recursively into 
subgroups such that an (approximately) homogeneous distributional model can be fitted to the response 
in each resulting subgroup.
To capture dependence on covariates, the association between the model's scores and each available
covariate is assessed using either a parameter instability test (MOB) or a permutation test (CTree).
In each partitioning step, the covariate with the highest significant association (i.e., lowest
significant $p$-value, if any) is selected for splitting the data. The corresponding split point
is chosen either by optimizing the log-likelihood (MOB) or a two-sample test statistic (CTree)
over all possible partitions.

In this study distributional trees are adapted to circular responses by employing the von Mises
distribution, also known as ``the circular normal distribution''. Based on a location parameter
$\mu \in \mathbb{R}$ and a concentration parameter $\kappa > 0$ the density for $y \in [0, 2 \pi]$
is given by:
\begin{equation}
  f_\mathrm{vM}(y; \mu, \kappa) = \frac{1}{2 \pi I_0(\kappa)}~e^{ \kappa \cos(y - \mu)}\label{equ:vm}
\end{equation}
where $I_0(\kappa)$ is the modified Bessel function of the first kind and order $0$
(see, e.g., Jammalamadaka and Sengupta (2001) for a more detailed overview).

In each subgroup maximum likelihood estimators $\hat \mu$ and $\hat \kappa$ are obtained
by maximizing the corresponding log-likelihood $\ell(\mu, \kappa; y) = \log(f_\mathrm{vM}(y;\mu, \kappa))$. 
The model scores are calculated by  $s(y; \mu, \kappa) = (\partial_{\mu} \ell(\mu, \kappa; y),
\partial_{\kappa} \ell(\mu, \kappa; y))$. In a subgroup of size $n$, evaluating the scores
at the individual observations and parameter estimates $s(y_i; \hat{\mu}, \hat{\kappa})$
yields an $n \times 2$ matrix that can be employed as a kind of residual, capturing how well
a given observation conforms with the estimated location $\hat{\mu}$ and precision $\hat{\kappa}$, 
respectively.
Hence MOB can test for parameter instabilities in the model by assessing whether the scores
change along with the available covariates. If so, by maximimizing a partitioned likelihood
the parameter instabilities are incorporated into the model. This procedure is repeated recursively
until there are no significant parameter instabilities or until another stopping criterion
is met (e.g., concerning subgroup sizes or depth of the tree).


\section{Application}

\begin{figure}[!ht]\centering
\includegraphics[height = .5\textheight,angle=90,origin=c]{circtree_plot_big.pdf}

\caption{Fitted distribution tree for a circular response employing the von
Mises distribution. For each terminal node, the empirical histogram and the
fitted density distribution is displayed in gray and red, respectively. The
location parameters are depicted as red hands and convey the direction from
which the wind is blowing. The chosen split variables are wind direction
(dd~[$\text{rad}$]), wind speed (ff~[$\text{ms}^{-1}$]), and pressure gradients
along the upper (pdiff\_up~[$\text{hPa}$]) and lower
(pdiff\_low[$\text{hPa}$]) Inn Valley where positive values point in the
direction of Innsbruck.}
\label{fig:tree} \end{figure}

Wind is a classical circular quantity and accurate forecasts of wind direction
are of great importance for decision making processes and risk management,
e.g., in air traffic management or renewable energy production. This study
presents an application of circular regression trees  on hourly direction
forecasts at Innsbruck Airport. Innsbruck lies at the bottom of the Inn
Valley, a deep mountain valley in the Eastern European Alps orientated parallel
to the Alpine crest. At Innsbruck, the west-east aligned Inn Valley is bordered
in the North with a approx. 2300\,m high mountain ridge and opens to the South
into the crest-perpendicular Wipp Valley. As potential split variables we
employ several 1-hourly lagged observed weather variables, amongst others, the
pressure gradient along the upper and lower Inn Valley, the wind direction, and
the wind speed at Innsbruck airport.

As an illustration of the tree based wind direction forecasts at Innsbruck
airport, a regression tree of depth three is depicted in Figure~\ref{fig:tree}
showing the empirical histogram and fitted response distribution for each terminal node.
According to the fitted location parameters the following wind regimes can be summarized:
Firstly, up-valley winds blowing from the valley mouth towards the upper valley
(from east to west, nodes 4 and 5). Secondly, downslope winds blowing across
the Alpine crest along the Wipp Valley towards Innsbruck (from south-east to
north-west, nodes 7 and 8). Thirdly, down-valley winds blowing in the direction
of the valley mouth (from west to east, nodes 12, 14, and 15). The terminal node
11 with a concentration parameter $\kappa$ of only $0.84$ captures the cases
with rather low wind speeds which typically do not show a clear wind direction
or distinct wind regime. In general, winds typically blow from areas of high
pressure to areas of low pressure where strong pressure gradients yield high
wind speeds and a rather well-defined wind direction. Thus, the splits in the
lagged wind direction perform a clustering into the different wind regimes
employed by the fitted location parameters, and the additional splits in
pressure gradients or wind speed divide the predictions according to the
expected consistency of the wind direction expressed by the fitted
concentration parameters. This can be seen in the terminal nodes with strong
pressure gradients or high wind speeds where most training cases are classified
correctly and therefore the concentration parameter as an indicator of the
certainty of the forecast is rather large. 

\section{Discussion and Outlook}
This study shows that applying distributional trees on circular data provides easily 
interpretable models clearly illustrating the impact of given covariates and their possible 
interactions on the distribution of the dependent variable. 
Moreover, covariates are selected automatically such that no prespecification of the model is 
required which can be a challenging task, particularly for a meteorlogical quantity such as wind 
direction with high variation and many possible influencing factors.
%Another valuable feature is the capability of capturing also non-additive and abrupt changes, 
Additionally, abrupt changes are captured, e.g., when the impact of a covariate such as 
\textit{air pressure difference} determines either up- or down-valley wind directions.

A possible extension of the method as presented is to build an ensemble of 
distributional trees to regularize and stabilize the model 
(see, e.g., distributional forests, Schlosser et.~al, 2018). 
%In that way, it should not have a 
%significant impact on the results whether circular covariates are included as linear regressors
%yielding a larger number of splits or as circular regressors with ``intervall splits''.

Note that the circular variable $dd$ (lagged observations of wind) is employed as a linear
regressor. Therefore, several splits are performed in $dd$ in order to properly cluster the training 
data in corresponding wind directions (see Figure~\ref{fig:tree}). 
In this case searching for two split points defining a split interval on the circular scale instead 
of a single split point on the linear scale could reduce the complexity of the model and possibly 
improve the results.
%Alternatively, in the case of a circular covariate 
%the tree building algorithm could search for two split points defining a split interval instead of 
%a single split point. This would reduce the complexity of the model and possibly improve the results.
%Based on the results it can be assumed that including covariates such as lagged observations of 
%wind direction as linear regressors provides valuable information. 
%However, in order to properly account not only for the circular properties of the response but 
%also of regressors, additional adaptions of distributional trees are required. In particular, 
%if the selected split variable is a circular variable, e.g. wind direction, this demands for finding
%two split point defining a ``split interval'' instead of a single split point.
%This is also supported by the results illustrated in Figure~\ref{fig:tree} as mutliple splits in 
%the linearly employed regressor \textit{dd} (lagged wind direction) are required in order to 
%cluster the training data in corresponding wind directions.  
\textit{Computational Details:} The employed \textsf{R} packages are available on \textsf{R}-Forge at
https://R-Forge.R-project.org/projects/partykit/.


%While the application of the von Mises distribution allows for modeling a circular response variable,
%additional adaptions have to be made when including circular regressor variables in a distributional tree
%model. In particular, if the selected split variable is a circular variable this demands for finding
%not only one split point but a ``split interval''. This consideration is of interest for many applications,
%for example when modeling wind directions it might be desirable to include day time or earlier 
%observations of wind directions as regressor variables.


\bigskip


%***********************************************************************

% Tables can be included at any place in the text.
% As general format, use tables with horizontal lines only
% (i.e., no vertical lines separating columns).
% Start and end each table with a double horizontal line.
% Tables are incorporated using a table environment:

%\begin{table}[!ht]\centering
% A caption is put above the table and a label is defined
% to be used for reference to this specific table.
% Use labels which are very unlikely to be used by
% other contributors; for example, use labels starting
% with the surname of the first author.
%\caption{\label{smith:tab1} Caption text \textbf{ABOVE} the table.}
% A table with three columns, where the first is left aligned,
% the second has centred entries, and the third is right aligned,
% is generated as follows (note: if requested, use \cmidrule{} instead of \cline{})
%\medskip
%\begin{tabular}{lcr}
% First line:
%\toprule[0.09 em]
% The body of the table:
%Title col1 & Title col2 & Title col3 \\
%\midrule
%row1 col1  & row1 col2  & row1 col3  \\
%row2 col1  & row2 col2  & row2 col3  \\ %
%row3 col1  & row3 col2  & row3 col3  \\
% last line:
%\bottomrule[0.09 em]
%\end{tabular}
%\end{table}

% In the text, reference to the Table can be made as follows:
%We refer to Table~\ref{smith:tab1} for a summary of our main results. Have a look to Table~\ref{smith:tab2} for
%an additional example.

%\begin{table}[!ht]\centering
%\caption{\label{smith:tab2} Caption text \textbf{ABOVE} the table.}
%\medskip
%\begin{tabular}{lcr}
% First line:
%\toprule[0.09 em]
% The body of the table:
%  &\multicolumn{2}{c}{Title  for cols 2 -3} \\
%\cmidrule{2-3} %
%Title1 & Title2 & Title3 \\
%\midrule
%& $a$  & $c$ \\
%& $b$  & $d$ \\ %
%\midrule[0 em]
%Total  & $a+b$  & $n$  \\
% last line:
%\bottomrule[0.09 em]
%\end{tabular}
%\end{table}



%***********************************************************************

% Figures can be included at any place in the text.
% The only allowed formats for figures are pdf files.
%
% Please, do not include figures in any other format.
%
% Use file names which are very unlikely to be used by
% other contributors; for example, use file names starting
% with the surname of the first author.
% Figures are incorporated using a figure environment:
% Make sure you specify the extension of the file (pdf)

%Finally a figure (in \verb|.pdf|!)

%\begin{figure}[!ht]\centering
% You can pre-specify the width of the graph:
%\includegraphics[width=8cm]{circmax-004.pdf}
% Below the figure, a caption is put, and a label is defined
% to be used for reference to this specific figure.
% Use labels which are very unlikely to be used by
% other contributors; for example, use labels starting
% with the surname of the first author.
%\caption{\label{fig:circtree} Fitted distribution tree for a circular response employing the von Mises distribution.}
%\end{figure}


% In the text, reference to the Figure can be made as follows:
%We refer to Figure~\ref{smith:fig1} for a~graphical representation.


%***********************************************************************

% Acknowledgments, if needed:
%\acknowledgments{Special Thanks to ... }

%***********************************************************************

% References should be placed in the text in author (year) form.
% The list of references should be placed below IN ALPHABETICAL ORDER.
% (Please follow the format of the examples very tightly).

\references
\begin{description}

%\item[Breiman, L., Friedman, J. H., Olshen, R. A., and Stone, C. J.] (1984).
%     Classification and Regression Trees.
%     {\it Wadsworth}.

\item[Hothorn, T., Hornik, K., and Zeileis, A.] (2006).
     Unbiased Recursive Partitioning: A Conditional Inference Framework.
     {\it Journal of Computational and Graphical Statistics}, {\bf 15}, 3,
     651\,--\,674. 

\item[Jammalamadaka, S. R., and Sengupta, A.] (2001).
     Topics in Circular Statistics.
     {\it World Scientific}, {\bf 5}. 

%\item[Mardia, K. V., and Jupp, P. E.] (2009).
%     Directional Statistics.
%     {\it John Wiley \& Sons}, {\bf 494}. 

%\item[Rigby, R. A., and Stasinopoulos, D. M.] (2005).
%     Generalized Additive Models for Location Scale and Shape (with Discussion).
%     {\it Journal of the Royal Statistical Society C}, {\bf 54}, 3,
%     507\,--\,554.

\item[Schlosser, L., Hothorn, T., and Zeileis, A.] (2018).
     Distributional Regression Forests for Probabilistic Precipitation Forecasting in Complex Terrain.
     {\it arXiv.org E-Print Archive}, https://arxiv.org/abs/1804.02921v2.

\item[Zeileis, A., Hothorn, T., and Hornik, K.] (2008).
     Model-Based Recursive Partitioning.
     {\it Journal of Computational and Graphical Statistics}, {\bf 17}, 2,
     492\,--\,514. 
     

\end{description}

\end{document}
