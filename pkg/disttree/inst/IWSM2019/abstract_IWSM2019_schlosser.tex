
%***********************************************************************

% This is a template to be used for the preparation of
% papers submitted to the 34th International Workshop on
% Statistical Modelling, to be held in Guimar?es, Portugal,
% July 7-12, 2019.

% Please follow the following general guidelines:
%
% (1) Do not specify any definitions, commands or style parameters.
%     Upon submission, your file will be checked for presence of
%     \newcommand or \def statements and if found, error message will be reported
%     by the submission form.
%
% (2) Follow the template below very tightly.
%
% (3) Include .pdf figures using the \includegraphics
%      command, an example of which are given below.
%
% (4) Use file names which begin with the surname of the first author.
%
% (5) When creating labels for cross-references, please start always
%     by surname of the first author, e.g., \label{smith:likelihood}
%
% (6) The template below contains some example materials
%      to guide you through the preparation of your paper.  However,
%      remove all the redundant material from your final document
%      before submitting.

% The guidelines above are needed in order to be able to combine all
% the papers into a single proceedings book of acceptable quality.
% Please follow the guidelines as strictly as possible. Deviations may
% result in papers being either refused by the registration form
% or sent back to the authors with the request to change
% their documents according to the guidelines.

% Special characters:
% Please do not use special characters (e.g., accents).
% Use TeX composition instead, such as \~n, \'a, \`e, \v{s}, \r{u} etc.

% Changes as of IWSM 2013:
%  * \usepackage{booktabs} added which allows \toprule et al. in the tabular environment
%    (\hline\hline is not longer used)
%  * '^\T' added in iwsm.sty to denote transposed vectors and matrices within math (see example below)
%  * \usepackage{amsmath, amssymb} introduced since IWSM 2012 is allowed (allowing usage of boldsymbols
%    and other handy constructions (align, pmatrix etc.) within math)
%  * \usepackage{psfrag} introduced since IWSM 2012 is NOT allowed
%
%

%***********************************************************************
% PLEASE LEAVE THIS PART UNCHANGED
%***********************************************************************

\documentclass[twoside]{report}
\usepackage{iwsm}
\usepackage{graphicx}
\usepackage{amsmath, amssymb}
\usepackage{booktabs}

% Please do not specify any new definitions, any new commands,
% and do not specify any style parameters.
% The preamble of the document should be left unchanged.

\begin{document}

%***********************************************************************
% PLEASE INSERT YOUR CONTENT FROM HERE
%***********************************************************************

% Title and running title to be used as left header:
\title{Distributional Trees for Circular Data}
\titlerunning{Distributional Trees for Circular Data}

% Authors and running list of authors to be used as right header:
\author{Lisa Schlosser\inst{1}, 
Moritz N. Lang\inst{{1,2}},
Torsten Hothorn\inst{3},
Georg~J.~Mayr\inst{2},
Reto Stauffer\inst{1},
Achim Zeileis\inst{1}}
\authorrunning{Schlosser et al.}    %% use \authorrunning{Surname 1} if only 1 author
                                    %% use \authorrunning{Surname 1 and Surname2} if two authors
                                    %% use \authorrunning{Surname 1 et al.} if more than two authors

%% Institutes of all authors
%% Include city and country of each institute, do not include the full address.
%\institute{Universit\"at Innsbruck, Austria
%\and Universit\"at Z\"urich, Switzerland}

%%%%%%%% IF DEPARTMENTS ARE LISTED AS WELL CHANGE NUMBERING!!!  %%%%%%%%%%%%%%
\institute{Department of Statistics, Universit\"at Innsbruck, Innsbruck, Austria
\and Department of Atmospheric and Cryospheric Science, Universit\"at Innsbruck, Innsbruck, Austria
\and Epidemiology, Biostatistics and Prevention Institute, Universit\"at Z\"urich, Z\"urich, Switzerland}


% E-mail of presenting author for correspondence
\email{Lisa.Schlosser@uibk.ac.at}

% Brief abstract of the paper:
\abstract{
For probabilistic modeling of circular data the von Mises distribution is widely used.
To capture how its parameters change with covariates, a regression tree model
is proposed as an alternative to more commonly-used additive models. The resulting
distributional trees are easy to interpret, can detect non-additive effects, and select
covariates and their interactions automatically. For illustration, hourly wind 
direction forecasts are obtained at Innsbruck Airport based on a set of meteorological
measurements.
}


% Keywords (at most 5):
\keywords{Distributional Trees; Circular Response; Von Mises Distribution.}

% Produce the title:
\maketitle

%***********************************************************************

% Sections and subsections (do not use lower levels):

\section{Motivation}
%\textit{FIXME: should we include this section 'Motivation'?}

Circular data can be found in a variety of applications and subject areas, 
e.g., hourly crime rate in the social-economics, 
animal movement direction or gene-structure in biology, 
and wind direction as one of the most important weather variables in meteorology.
Circular regression models were first introduced by Fisher and Lee (1992) and 
further extended by Jammalamadaka and Sengupta (2001) and 
Mulder and Klugkist (2017) among others.
While most of the already existing approaches are built on additive regression models, 
we propose an adaption of regression trees to circular data by employing 
distributional trees.

%Representing such a two-dimensional direction by an angle in a unity circle requires 
%the specification of a starting point (angle 0) and either a clockwise or anti-clockwise 
%rotation. However, these settings can vary but should not influence statistical 
%inference from a fitted model.
%For that reason circular data demand for purpose-built statistical methods. 
%A commonly chosen distribution for modeling circular data is the von Mises distribution.
%Estimating its two distribution parameters by generalized additive models using a set of 
%regressor variables is one possible approach that can be successful in various settings.
%As an alternative we propose the application of distributional trees 
%which are capable of detecting also non-additive effects and provide easily interpretable 
%models allowing for a wide range of inference procedures.
%Moreover, covariates and their possible interactions are selected automatically
%such that no expert knowledge is required to specify the model in advance.

\section{Methodology}

Distributional trees~(Schlosser et.~al, 2019) fuse distributional regression modeling with
regression trees based on the unbiased recursive partitioning algorithms MOB~(Zeileis et.~al, 2008)
or CTree~(Hothorn et.~al, 2006). The basic idea is to partition the covariate space recursively into 
subgroups such that an (approximately) homogeneous distributional model can be fitted to the response 
in each resulting subgroup.
To capture dependence on covariates, the association between the model's scores and each available
covariate is assessed using either a parameter instability test (MOB) or a permutation test (CTree).
In each partitioning step, the covariate with the highest significant association (i.e., lowest
significant $p$-value, if any) is selected for splitting the data. The corresponding split point
is chosen either by optimizing the log-likelihood (MOB) or a two-sample test statistic (CTree)
over all possible partitions.

In this study distributional trees are adapted to circular responses by employing the von Mises
distribution, also known as ``the circular normal distribution''. Based on a location parameter
$\mu \in [0, 2\pi]$ and a concentration parameter $\kappa > 0$ the density for $y \in [0, 2\pi]$
is given by:
\begin{equation}
  f_\mathrm{vM}(y; \mu, \kappa) = \frac{1}{2 \pi I_0(\kappa)}~e^{ \kappa \cos(y - \mu)}\label{schlosser:equ_vm}
\end{equation}
where $I_0(\kappa)$ is the modified Bessel function of the first kind and order $0$
(see, e.g., Jammalamadaka and Sengupta 2001, for a more detailed overview).

In each subgroup maximum likelihood estimators $\hat \mu$ and $\hat \kappa$ are obtained
by maximizing the corresponding log-likelihood $\ell(\mu, \kappa; y) = \log(f_\mathrm{vM}(y;\mu, \kappa))$. 
The model scores are given by  $s(y; \mu, \kappa) = (\partial_{\mu} \ell(\mu, \kappa; y),
\partial_{\kappa} \ell(\mu, \kappa; y))$. In a subgroup of size $n$, evaluating the scores
at the individual observations and parameter estimates $s(y_i; \hat{\mu}, \hat{\kappa})$
yields an $n \times 2$ matrix that can be employed as a kind of residual, capturing how well
a given observation conforms with the estimated location $\hat{\mu}$ and precision $\hat{\kappa}$, 
respectively.
Hence MOB or CTree can assess whether the scores change along with the available covariates.
If so, by maximizing a partitioned likelihood
the parameter instabilities are incorporated into the model. This procedure is repeated recursively
until there are no significant parameter instabilities or until another stopping criterion
is met (e.g., subgroup size or tree depth).


\section{Application}

%% exchange file name of figure: schlosser-circtree_plot.pdf -> schlosserfig1.pdf
\begin{figure}[p!]\centering
\includegraphics[height = .5\textheight,angle=90,origin=c]{schlosser-circtree_plot.pdf}
\caption{Fitted tree based on the von Mises distribution for wind direction forecasting.
In each terminal node the empirical histogram (gray) and fitted density (red line)
are depicted along with the estimated location parameter (red hand). The covariates
employed are wind direction (degree), wind speed ($\text{ms}^{-1}$),
and pressure gradients ($\text{dpressure; hPa}$) west and east of the airport,
all lagged by one hour.}
\label{schlosser:fig_tree} \end{figure}

Wind is a classical circular quantity and accurate forecasts of wind direction
are of great importance for decision-making processes and risk management,
e.g., in air traffic management or renewable energy production. This study
employs circular regression trees to obtain hourly wind direction
forecasts at Innsbruck Airport.
Innsbruck lies at the bottom of a deep valley in the Alps. Topography
channels wind along the west-east valley axis or along a tributary valley
intersecting from the south. Hence, pressure gradients to which valley wind
regimes react both west and east of the airport are considered as covariates
along with other meteorological measurements at the airport (lagged by one hour),
such as wind direction and wind speed at Innsbruck Airport.
Note that in the meteorological context wind direction is defined on the scale $[0,360]$ degree and increases clockwise from North ($0$ degree).

Figure~\ref{schlosser:fig_tree} depicts the resulting distributional tree, including
both the empirical (gray) and fitted von Mises (red) distribution of wind direction
in each terminal node. Based on the fitted location parameters $\hat \mu$, the subgroups
can be distinguished into the following wind regimes:
(1)~Up-valley winds blowing from the valley mouth towards the upper valley
(from east to west, nodes 4 and 5). (2) Downslope winds blowing across
the Alpine crest along the intersecting valley towards Innsbruck (from south-east to
north-west, nodes 7 and 8). (3) Down-valley winds blowing in the direction
of the valley mouth (from west to east, nodes 12, 14, and 15). Node~11
captures observations with rather low wind speeds that cannot be distinguished
clearly into wind regimes and consequently are associated with a very low
estimated concentration $\hat \kappa$. In terms of covariates, the lagged
wind direction (``persistence'') is mostly responsible for distinguishing
the broad wind regimes listed above while the pressure gradients and wind
speed separate between subgroups with high vs.\ low precision.

\section{Discussion and outlook}

Distributional trees for circular responses are established by coupling
model-based recursive partitioning with the von Mises distribution.
The resulting trees can capture nonlinear changes, shifts, and potential interactions
in covariates without prespecification of such effects. This is particularly
useful for modeling wind direction in mountainous terrain where wind shifts
can occur due to turns of the pressure gradients along a valley.

\subsection{Ensembles and random forests}
A natural extension are ensembles or forests of such circular trees
that can improve forecasts by regularizing and stabilizing the model.
Random forests introduced by Breiman~(2001) average the
predictions of an ensemble of trees, each built on a subsample 
or bootstrap of the original data. 
A generalization of this strategy is to obtain weighted predictions
by adaptive local likelihood estimation of the distributional parameters
(Schlosser~et.~al, 2019). More specifically, for each possibly new
observation~$x$ a set of ``nearest neighbor'' weights $w_i(x)$ is obtained 
that is based on how often $x$ is assigned to the same terminal node as
each learning observation $y_i, i \in \{1,\ldots,n\}$.

The parameters $\mu$ and $\kappa$ are then estimated for each (new) 
observation $x$ by weighted maximum likelihood based on the
adaptive nearest neighbor weights:
\begin{equation}
\operatorname{argmax}\displaylimits_{\mu, \kappa} \sum_{i=1}^n w_i(x) \cdot \ell(\mu, \kappa; y_i). 
\end{equation}
Therefore, the resulting parameter estimates can smoothly adapt to the given
covariates $x$ whereas $w_i(x) = 1$ would correspond to the unweighted
full-sample estimates and $w_i(x) \in \{0, 1\}$ corresponds to the abrupt
splits from the tree.

\subsection{Splits in circular covariates}

In order to obtain more parsimonious and more stable trees another possible
extension for \emph{circular covariates} (with or without a \emph{circular response})
is to consider their circular nature when searching the best split into two segments.
In general, searching the best separation of a covariate into a ``left'' and ``right''
daughter node tries to maximize the segmented log-likelihood:
\begin{equation}
\max \left(\sum_{y \in \mathit{left}} \ell(\hat{\mu_1}, \hat{\kappa_1}; y) + \sum_{y \in \mathit{right}} \ell(\hat{\mu_2}, \hat{\kappa_2}; y)\right)
\end{equation}
where $\hat{\mu_1}$, $\hat{\kappa_1}$, $\hat{\mu_2}$, $\hat{\kappa_2}$ are the estimated parameters
of the von Mises distribution in the two daughter nodes. Searching a single split point $\nu$ in a circular covariate $\in [0, 2 \pi)$
only considers linear splits into the intervals $\mathit{left}=[0,\nu]$ and $\mathit{right}=(\nu,2\pi)$,
thus enforcing a potentially unnatural separation at zero. This can be avoided by searching
for two split points $\nu$ and $\tau$ considering a split into one interval $\mathit{left}=[\nu,\tau]$
and its complement $\mathit{right}=[0,\nu) \cup (\tau,2\pi]$, encompassing zero. The latter
strategy is invariant to the (often arbitrary) definition of the direction at zero.

When one split point $\nu$ is sufficiently close to zero and the other $\tau$ sufficiently
far away, a simple linear split typically suffices to capture such a split (as seen for the
lagged wind direction in Figure~\ref{schlosser:fig_tree}). If both $\nu$ and $\tau$ differ
clearly from zero, two linear splits should also lead to a reasonable (but less parsimonious)
fit. However, if both $\nu$ and $\tau$ are rather close to zero, a linear split strategy
might miss such a pattern.

The required test statistic to maximally select two split points simultaneously is straightforward
to accommodate in the CTree framework by providing all binary indicators corresponding
to the splits into $\mathit{left}$/$\mathit{right}$ intervals. However, this will become
increasingly slow for larger sample sizes but it might be possible to speed up computations by
exploiting the particular covariance structure similar to Hothorn and Zeileis~(2008). In the
MOB framework the extension is not quite as straightforward but one strategy could be to
adapt double maximum tests \`a la Bai and Perron~(2003).

Hence, the splitting idea can be naturally extended to a two-point search, however, for 
an unbiased and inference-based selection the corresponding testing strategies might need
further adaption.

\bigskip

\subsubsection*{Computational details:}
\textsf{R} packages implementing the proposed methods are currently under development at
\texttt{https://R-Forge.R-project.org/} \texttt{projects/partykit/}.


% Acknowledgments, if needed:
\acknowledgments{This project was partially funded by the Austrian Research Promotion 
Agency~(FFG) grant no.~$858537$.}

\bigskip

\references
\begin{description}

\item[Bai, J., and Perron, P.] (2003).
     Computation and Analysis of Multiple Structural Change Models.
     {\it Journal of Applied Econometrics}, {\bf 18},
     1\,--\,22.     
     
\item[Breiman, L.] (2001).
     Random Forests.
     {\it Machine Learning}, {\bf 45}, 1,
     5\,--\,32.

\item[Fisher, N. I., and Lee, A. J.] (1992).
     Regression Models for an Angular Response.
     {\it Biometrics}, {\bf 48}, 3,
     665\,--\,677.  

\item[Hothorn, T., Hornik, K., and Zeileis, A.] (2006).
     Unbiased Recursive Partitioning: A Conditional Inference Framework.
     {\it Journal of Computational and Graphical Statistics}, {\bf 15}, 3,
     651\,--\,674. 
     
\item[Hothorn, T., and Zeileis, A.] (2008).
     Generalized Maximally Selected Statistics.
     {\it Biometrics}, {\bf 64}, 4,
     1263\,--\,1269. 

\item[Jammalamadaka, S. R., and Sengupta, A.] (2001).
     {\it Topics in Circular Statistics}.
     World Scientific. 

\item[Mulder, K., and Klugkist, I.] (2017).
     Bayesian Estimation and Hypothesis Tests for a Circular Generalized Linear Model.
     {\it Journal of Mathematical Psychology}, {\bf 80},
     4\,--\,14. 

%\item[Mardia, K. V., and Jupp, P. E.] (2009).
%     Directional Statistics.
%     {\it John Wiley \& Sons}, {\bf 494}. 

%\item[Rigby, R. A., and Stasinopoulos, D. M.] (2005).
%     Generalized Additive Models for Location Scale and Shape (with Discussion).
%     {\it Journal of the Royal Statistical Society C}, {\bf 54}, 3,
%     507\,--\,554.

\item[Schlosser, L., Hothorn, T., Stauffer, R., and Zeileis, A.] (2019).
     Distributional Regression Forests for Probabilistic Precipitation Forecasting in Complex Terrain.
     arXiv:1804.02921, {\it arXiv.org E-Print Archive}.

\item[Zeileis, A., Hothorn, T., and Hornik, K.] (2008).
     Model-Based Recursive Partitioning.
     {\it Journal of Computational and Graphical Statistics}, {\bf 17}, 2,
     492\,--\,514. 
     
\end{description}

\end{document}
