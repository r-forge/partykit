\documentclass[english, noconfig]{uibklttr}

\newcommand{\section}[1]{{\Large{\textbf{#1}}}}
\newcommand{\subsection}[1]{{\large \textbf{#1}}}
\newcommand{\subsubsection}[1]{{\large \textbf{#1}}}
\newcommand{\tightlist}[0]{}

% Some colors
\definecolor{myblue}{rgb}{0.2,0.2,0.9}
\definecolor{mygray}{rgb}{0.7,0.7,0.7}

\makeatletter
\newenvironment{thebibliography}[1]
     {\list{\@biblabel{\@arabic\c@enumiv}}%
           {\settowidth\labelwidth{\@biblabel{#1}}%
            \leftmargin\labelwidth
            \advance\leftmargin\labelsep
            \usecounter{enumiv}%
            \let\p@enumiv\@empty
            \renewcommand\theenumiv{\@arabic\c@enumiv}}%
      \sloppy
      \clubpenalty4000
      \@clubpenalty \clubpenalty
      \widowpenalty4000%
      \sfcode`\.\@m}
     {\def\@noitemerr
       {\@latex@warning{Empty `thebibliography' environment}}%
      \endlist}
\newcommand\newblock{\hskip .11em\@plus.33em\@minus.07em}
\makeatother
\usepackage{natbib}

%\usepackage{subcaption}
%\captionsetup{font={bf,small},skip=0.25\baselineskip}
%\captionsetup[subfigure]{font={bf,small}, skip=1pt, singlelinecheck=false}

\usepackage{amsmath}

\usepackage{float} % For figure placements
\usepackage{enumitem}
\setitemize{itemsep=0em}
\setlist{itemsep = 0em}

\usepackage{parskip}
\setlength{\parskip}{.4em}

% fake-float with caption for figures
\usepackage{caption}
\usepackage{newfloat}
\DeclareFloatingEnvironment[fileext=lot]{figure}
\DeclareCaptionFont{myblue}{\color{myblue}}
\captionsetup{labelfont={bf},textfont={it}}

% short re
\newcommand{\sre}[1]{
    \begin{color}{uibkgraym}#1\end{color}%
}
% long re
\newenvironment{re}{
    \begin{color}{uibkgraym}
        \itshape
}{
    \end{color}
}
\newcommand{\todo}[1]{
    \begin{color}{red}\textbf{TODO:} #1\end{color}%
}
\newcommand{\note}[1]{
    \begin{color}{orange}\textbf{Reto's comment:} #1\end{color}%
}
\newcommand{\see}[2]{Page \textbf{#1}, line \textbf{#2}.}
\newcommand{\msee}[3]{Page \textbf{#1}, lines \textbf{#2--#3}.}
\newcommand{\fig}[1]{\textit{now Fig~#1}}
\newcommand{\nsec}[1]{\textit{now Sec~#1}}


\begin{document}

\setkomavar{subject}{Revision of `npg-2019-49'}
\setkomavar{signature}{Moritz Lang\\Corresponding Author}

\setkomavar{fromname}[Name]{Moritz Lang}
\setkomavar{fromemail}{Moritz.Lang@uibk.ac.at}




\setkomavar{fromaddress}{University of Innsbruck, Department of Statistics, Universit\"atsstrasse
15, 6020 Innsbruck, Austria}

%\begin{letter}[fromemail]{To Maxime Taillardat}
%\opening{Dear Christopher Paciorek}
%
%We thank you, as well as both reviewers, for the detailed and fruitful feedback
%regarding our manuscript ``\textbf{Bivariate Gaussian models for wind vectors in a
%distributional regression framework}''.
%
%\vspace{5mm}
%
%We have carefully revised our manuscript according to your comments and
%suggestions. The most substantial changes are the following:
%
%\begin{itemize}
%\item
%An additional comparison to the work of \citet{schuhen.etal:2012} and of
%\citet{pinson:2012} has been included. Both comparisons nicely confirm our
%previous results. To keep the manuscript as clear as possible, we describe the
%comparisons in the appendix and provide short summaries of the main finding
%in the discussion section.
%
%\item
%The introduction of the logarithmic score has been
%extended. We now explicitly derive the employed formulation of the logarithmic
%score and provide additional references.
%
%\item
%We have added more information on the estimation procedure and describe the key
%steps involved in the model fitting. Please also remark that for consistency we
%have changed `mean and variance` to `location and scale parameter` within the
%abstract and the introductory part of the discussion.
%\end{itemize}
%
%\vspace{5mm}
%
%On the following pages a point-to-point response to both reviewers and to the
%editor will be given. The attached manuscript highlights the changes in the
%text in blue color.
%
%\closing{Your sincerely,}
%
%\end{letter}
%
%\newpage

\section{Editor: Prof. Peter Smith - Comments to the Author}

\begin{itemize}

\item Please can you rewrite the Summary to give the application more prominence.
Currently, the last sentence reads to me like an afterthought. Furthermore, I
would prefer you to use the term ‘application’ rather than ‘case study’
throughout the paper, since in Applied Statistics papers the methodological
developments should be motivated by the application.

\item Page 2, line 46: ‘… structure can capture non-additive …’?

\item Page 2, lines 54 and 55: ‘… mean, whereas modelling the conditional … distribution would also allow uncertainty in the forecast to be estimated.’?

\item Page 6, line 16 and Equation (2.6): bold the vector $\beta$.

\item Page 6, lines 29 and 30: ‘In general, for additive models a proper model …’?

\item Equations (3.1) to (3.4) and elsewhere: I find the use of multiple letters for a single superscript or subscript awkward and prefer to keep lower case letters for indices, so please consider replacing them, for example, T for tree and F for forest. Also for Equation (4.2), how about using parentheses rather than subscripts?

\item Equation (3.4): bold vector $z$.

\item Page 9, line 39: delete ‘respectively’.

\item Page 15, line 23: replace ‘leveraged’ with ‘used’?

\end{itemize}

\newpage

\section{Associate Editor - Comments to the Author}

Both referees agree this is a good paper, and have only a few suggestions for
changes, so I recommend minor revisions to address these.

I have two short suggestions in addition to this:

\begin{itemize}
\item Can you comment on the typical computational requirements of the various
methods used in your application, in terms of hardware required and length of
computing time.

\item Is it possible to include a link to computer code to reproduce your
analysis? This is not essential for publication but would be helpful for
readers who wish to perform similar analyses. The R package you have provided
for your method is very helpful in this regard.  
\end{itemize}

\newpage

\section{Referee: 1 - Comments to the Author}

I found Circular Regression Trees and Forests to be a very nicely written
paper, with a meaningful contribution to the literature on regression models
for a circular response variable. 

I have very few suggestions to the authors; but here are two:

\begin{itemize}
\item On page 2, what is the reference for the phrase "Additionally, they
developed a combined ... version ...".

\item On page 11, regarding "As the implementation cannot handle circular
covariates ...": It could handle circular covariates in a fashion similar to
Lund (1999), using a rotational or reflective relationship, or modeling the
association with a trigonometric polynomial. Adding this to their comparison
model, or at least mentioning the possibility of doing so, would be more
appropriate than stating that circular covariates cannot be accommodated.
\end{itemize}

\vspace{0.5cm}
References

Lund UJ (1999). ``Least Circular Distance Regression for Directional Data.''
\emph{Journal of Applied Statistics}, \textbf{26}\,(6), 723--733.
doi:10.1080/02664769922160.

\newpage

\section{Referee: 2 - Comments to the Author}

This is a well-written paper on using circular regression trees for analyzing a
specific type of circular data. The probabilistic approach is well-founded. The
experiments on wind direction data are well organized and the results are
thoroughly analyzed.

\begin{itemize}
\item The authors could address model order determination problem using the
Bayes Information Criterion (Schwartz, 1978, Annals of Statistics).  

\item I would like to point out related works on circular regression models for
shape representation (Kashyap and Chellappa) and circular random forests for
vehicle direction estimation (Hara and Chellappa) given below. I feel there is
some overlap between this paper and the work of Hara and Chellappa on using
circular statistics for vehicle orientation estimation. The authors may include
some comparisons with Hara, Chellappa work.
\end{itemize}

\vspace{0.5cm}
References

Hara K, Chellappa R (2017). ``Growing Regression Tree Forests by
Classification for Continuous Object Pose Estimation.'' \emph{International
Journal of Computer Vision}, \textbf{122}\,(2), 292--312.
doi:10.1007/s11263-016-0942-1.

Kashyap R, Chellappa R (1981). ``Stochastic Models for Closed Boundary
Analysis: Representation and Reconstruction.'' \emph{IEEE Transactions on
Information Theory}, \textbf{27}\,(5), 627--637. doi:10.1109/TIT.1981.1056390.

\end{document}
